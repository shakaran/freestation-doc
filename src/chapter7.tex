% -*- coding: utf-8 -*-
%%\mainmatter
\cleardoublepage
\chapter{Conclusiones y propuestas}
\thispagestyle{fancy}
%%\setcounter{page}{94}

\section{\uppercase{Objetivos alcanzados}}

Partiendo de los requerimientos iniciales, el enfoque aplicado propuesto en este
proyecto de fin de carrera alcanza los resultados previstos.

La aplicación tiene una funcionalidad completa en base a la especificación de
objetivos inicial. Se ha desarrollado por completo el concepto origen del
proyecto de fin de carrera y se han utilizado todas las tecnologías inicialmente previstas. Esto ha sido posible gracias a
la independencia de un modelo específico y la
reusabilidad de bibliotecas y aplicaciones existentes.

En el lado cliente se han finalizado los principales módulos esenciales como
Watchdog, tratamiento de un conjunto suficientemente amplio de widgets y el
módulo de comunicación con el servidor.

El sistema permite la generación dinámica de interfaces complejas según los
requisitos especificados por el cliente de forma precisa. Los datos son
fácilmente exportables con configuraciones basadas en un esquema XML.

Gracias a este enfoque se consigue un alto nivel de adaptabilidad, permitiendo
la futura realización de mejoras internas, adaptándose al desarrollo de
bibliotecas y aplicaciones de terceros. Por otro lado, el componente servidor
funciona perfectamente con la conexión de backend a clientes mediante \acs{ICE} y utiliza el frontend web
basado en PHP facilitando la administración web.

\newpage

Otro de los aspectos a destacar es la incorporación de nuevas tecnologías como
CouchDB (ver sección ~\ref{sec:couchdbapache}), microfiber y widgets basados en
HTML 5.

El desarrollo del proyecto no sólo supondrá un beneficio directo para otros
desarrolladores con similares propuestas, sino para usuarios que usen la
aplicación a diario. En cualquier caso, es preciso tener en cuenta que en este
proyecto de fin de carrera se han realizado casos funcionales de explotación a 
modo de demostración del mismo. \emph{FreeStation} se ha desarrollado teniendo
en cuenta su posición como herramienta útil, que permita el manejo y la gestión 
de grandes volúmenes de información.

Como parte de su objetivo se han tomado medidas de simplificación en tareas
complejas para el usuario facilitando el proceso de interacción del mismo a
través del diseño de interfaces naturales (NUI) (ver sección
~\ref{sec:interfazNUI}). Gran parte de estas interfaces son generadas
dinámicamente siguiendo una especificación definida por la aplicación.

Todo los objetivos y subobjetivos han sido cumplidos adecuadamente. Una de las
partes clave en \emph{FreeStation} es la adaptabilidad de para los requisitos
de una organización. La robustez del sistema permite la recuperación de errores mediante el
subsistema de \emph{Watchdog} visto en la sección ~\ref{sec:watchdog}.

Las tecnologías que se utilizan en el proyecto de fin de carrera ayudan a
homogeneizar el software a través del diseño modular del sistema. Su
versatilidad permite la reutilización de componentes. El código de 
\emph{FreeStation} se basa en buenos principios de diseño basados en el uso 
de patrones implementando varios de ellos (ver sección ~\ref{sec:patrones}). El
sistema basado en programación distribuida permite cumplir los objetivos de un 
software actualizable y distribuido usando el framework ICE (ver sección
~\ref{sec:ZeroICE}). 

Por último, \emph{FreeStation} basa su desarrollo en software y estándares
libres lo que permite posible continuación a otras personas interesadas.

\newpage

\section{\uppercase{Propuestas de trabajo futuro}}

El desarrollo de FreeStation ha completado una arquitectura completa para la
definición dinámica de POIs específicos para la distribución de software. Sin
embargo, el proyecto tiene grandes ambiciones futuras y se proponen 
futuras propuestas que permitan mejorar y adaptar las herramientas a nuevos
desarrollos o incluso adaptarlas a las características específicas de un equipo
de trabajo.

A continuación se enumeran algunas de las líneas de trabajo futuras propuestas:

\begin{itemize}
     \item \textbf{Ampliación de widgets}:\\
     El desarrollo de widgets actuales puede ser ampliado y ofrecer un gran catálogo
    de widgets adicionales. Explotar la posibilidad de funcionalidades basadas
    en 3D con widgets basados en OpenGL e investigar widgets más útiles para
    el público general. Los widgets basados en sonidos de música ambientes o
    en combinación con efectos visuales pueden ser útiles para el modo idle de
    un POI. En definitiva, se pretende buscar formas de mejorar los widgets
    actuales y añadir widgets novedosos. Como ayuda se pueden realizar 
    encuestas a organizaciones para conocer sus necesidades prioritarias 
    en sus POIS particulares. 
    Estimación de tiempo: 2-3 meses
    
    \item \textbf{Sustitución del almacenamiento de datos para widgets}:\\
    El modelo de almacenamiento de datos utilizado es XML.
    Éste podría ser sustituido por una base de datos. El patrón MVC permite 
    modificar el modelo de datos sin afectar al resto de componentes. La
    información podría ser guardada en una base de datos sqlite debido a la poca
    necesidad de usuarios concurrentes en un mismo POI. Es de suma importancia
    la gestión que realiza con los datos y por tanto debe asegurarse las mismas
    condiciones de fiabilidad y tiempo de procesado. 
    Estimación de tiempo: 1-2 meses
    
    \item \textbf{Incorporar un caso de explotación con localización de
    idiomas}:\\
    Como se comentó en la sección ~\ref{sec:selectlanguage} la incorporación de
    un selector de idiomas en un POI beneficia mucho su adaptación y permite llegar a
    un mayor número de personas.
    La realización de un widget o una adaptación sobre HTML4 y CouchDB (véase
    sección ~\ref{sec:couchdbapache}) sería modular para reutilización en otros
    POIs. Es necesario buscar un equilibrio en el número de idiomas ofrecidos al
    público del usuario. La estimación de 3 a 5 idiomas en la mayor parte de los
    casos podría ser util.
    Estimación: 1-2 meses.

\end{itemize}

\newpage

\section{\uppercase{Conclusión personal}}

La realización del proyecto de fin de carrera me ha permitido llevar a cabo un
proyecto real e investigación de nuevas tecnologías y diferentes enfoques de
resolución de problemas.

El tiempo destinado ha sobrepasado el inicialmente requerido, pero ello me
ha permitido desarrollar mejores conceptos y soluciones a través de iteraciones
progresivas en el mismo.

De la realización del proyecto se desprende el gran potencial que disponen
algunas nuevas tecnologías emergentes, en particular basadas en el ámbito
web y en combinación con las aplicaciones de escritorio regulares.

Gracias a que la gran mayoría del software
utilizado este proyecto ha sido basado en software libre se ha podido estudiar
profundamente su funcionamiento y incorporar mejoras y contribuciones cuando ha
sido posible a la comunidad.

El coste económico del proyecto es viable incluso para pequeñas instituciones o
empresas que requieran en un futuro el uso de las aplicaciones desarrolladas.
La labor de desarrollo afrontada compensa con la ganancia en productividad que
puede llegar a retornar a los usuarios.

Las aproximaciones de prototipos y posteriores desarrollos fueron los más
óptimos e inmediatos para comenzar a esbozar el proyecto, su ciclo posterior de
desarrollo ha tenido un mayor esfuerzo, pero ha permitido llevar a cabo una labor
proyecto completo de ingeniería resolviendo problemas.

Como conclusión final podemos afirmar que este proyecto satisface las especificaciones 
iniciales, pero deja abierto un amplio abanico de ampliaciones e implantaciones futuras.

%% Put a blank page for open next chapter on right page side
%%\newpage
%%\mbox{}
%%\thispagestyle{empty}
\cleardoublepage
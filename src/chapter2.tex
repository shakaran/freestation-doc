% -*- coding: utf-8 -*-
\cleardoublepage
\chapter{Objetivos}
\thispagestyle{fancy}
%%%\setcounter{page}{17}

\section{\uppercase{Objetivo principal}}

\drop{E}{l} objetivo principal de este proyecto fin de carrera es crear una
plataforma con la que construir, desarrollar y desplegar sistemas de
distribución de software libre de una organización adaptados a sus 
necesidades.

Los usuarios, a través de terminales de acceso con interfaces de manipulación
directa, podrán solicitar entre un catálogo de software disponible y realizar
operaciones para cumplir sus los objetivos de forma rápida y
segura.

Para la consecución de este objetivo funcional general, se han identificado una
serie de requisitos no funcionales que se resumen a continuación:

\begin{itemize}
    \item{Identificar y recoger los requisitos de la aplicación para dar
    comienzo a la fase de desarrollo.}
    \item{Realizar un estudio del panorama actual sobre distribución de software en
    puntos de interés.}
    \item{Analizar las diferentes tecnologías para la creación de una aplicación web:
    Entornos de desarrollo, bases de datos, lenguajes de programación,
    frameworks, bibliotecas y navegadores.}
    \item{Creación de la aplicación web con los requisitos anteriormente
    identificados y siguiendo la metodología de la XP.}
\end{itemize}

\newpage

\section{\uppercase{Subobjetivos}}

El objetivo general descrito anteriormente será alcanzado en base al 
cumplimiento de una serie de subobjetivos funcionales y no funcionales 
que se describen a continuación:
\begin{itemize}
  
    \item{\textbf{Adaptabilidad}. El sistema debe permitir trabajar con
    diferentes configuraciones con respecto al número de nodos de la organización, así 
    como de los requisitos particulares de cada uno de ellos. La diversidad de 
    requisitos deberá estar presente desde el inicio del proyecto, facilitando 
    la adaptación y ampliación a nuevas necesidades.}

    \item{\textbf{Robustez}. El sistema debe ser muy robusto, y deberá contar
    con mecanismos para el control y recuperación de errores, para garantizar su 
    correcta puesta en explotación en un entorno real\cite{Dea98}. La plataforma 
    facilitará el registro y la medición de datos de cada actividad desarrollada
    en el sistema, para prevenir inconsistencia y avisar de posibles fallos 
    producidos.}
  
   \item{\textbf{Heterogeneidad}. La plataforma se encargará de ofrecer una capa
    de homogeneización del software que abstraiga de las propiedades específicas
    del hardware utilizado en cada punto de información. El diseño modular del
    sistema permitirá su utilización de forma legible y manejable, facilitando 
    una correcta parametrización y reutilización de componentes. Un sistema
    extensible, además permitirá una modificación fácil de su comportamiento sin
    tener que realizar cambios en todo el sistema.}
    
    \item{\textbf{Distribuido}. El sistema se diseñará para minimizar el impacto
    en el ancho de banda en red (evitando la transmisión de información 
    redundante), permitiendo el despliegue de varios nodos, eliminando puntos 
    únicos de fallo (\acs{SPOF}\label{acro:SPOF}) mediante el uso de componentes
    y objetos distribuidos.}
    
    \item{\textbf{Diseño \acs{NUI}\label{acro:NUI}}. La plataforma estará
    orientada a la puesta en explotación de sistemas basados en el paradigma
    NUI, explotando de un modo eficiente las posibles capacidades de 
    subsistemas de vídeo, audio y otros contenidos interactivos. El sistema
    definirá un conjunto de widgets orientados a la manipulación directa.}
    
    \item{\textbf{Actualizable}. El sistema debe permitir la fácil 
    actualización en todos los nodos del a organización de un modo 
    automático. El impacto en tiempo para el usuario debe ser mínimo, de modo 
    que las actualizaciones o cambios en nuevas versiones del software deben
    realizarse de modo desatendido.}

    \item{\textbf{Libre}. La plataforma estará basada en estándares libres. En 
    el desarrollo del proyecto se utilizarán alternativas basadas en software 
    libre y uso de estándares consensuados.}
\end{itemize}

%% Put a blank page for open next chapter on right page side
%%\newpage
%%\mbox{}
%%\thispagestyle{empty}
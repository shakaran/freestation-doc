% -*- coding:utf-8 -*-

%% Para slantsc.sty
%% sudo apt-get install texlive-latex-extra
%% sudo apt-get install aspell-es

%% http://es.wikibooks.org/wiki/Manual_de_LaTeX/La_estructura_de_un_documento_en_LaTeX/Clases_de_documento_y_algunos_paquetes_comunes
\documentclass[a4paper, 12pt, twoside, openright,
 openany, titlepage, final]{book} %% cleardoublepage=plain,
\usepackage[left=35mm, right=20mm, top=25mm, bottom=25mm]{geometry}
\usepackage[utf8]{inputenc} % IMP! Para mantener acentos al seleccionar texto en visor de pdf
\usepackage[spanish,activeacute]{babel} %paquete necesario: textlive-lang-spanish (para guiones) y texconfig init >log
\usepackage[scaled]{helvet}
\usepackage{slantsc}
\usepackage{lmodern}
\usepackage[T1]{fontenc}
\usepackage{textcomp}
\usepackage{fix-cm} %% Avoid LaTeX Font Warning: Font shape
\usepackage{amssymb,amsmath}
\usepackage{latexsym}
\usepackage{relsize}
\usepackage{etoolbox}
\usepackage{bookman} % paquete necesario: textlive-font-recommended
%%\usepackage{times} % Usar Times New Roman
%% Control the fonts and formatting used in the table of contents.

\usepackage[titles]{tocloft}

%% -- ACRÓNIMOS Y GLOSARIO
\usepackage[printonlyused,withpage]{acronym} % acronym

\renewcommand*{\acsfont}[1]{\textsc{\textscale{.85}{#1}}} % enunciado del acrónimo: OO
\renewcommand*{\acfsfont}[1]{#1}
\renewcommand*{\acffont}[1]{#1}


\newcommand\subHalmos{\rule[0.2pt]{2.4pt}{2.4pt}}
\newcommand\Halmos{\rule[0.4pt]{4pt}{4pt}}

\makeatletter
  \def\uplabel#1{{\normalfont{$\Halmos$ \textsf{#1}}\hfill}-}
  %% Fist part
  \renewenvironment{AC@deflist}[1]%
    {\ifAC@nolist%
     \else%
        \raggedright\begin{list}{}%
            {
                \settowidth{\labelwidth}{\normalfont{\textsf{#1}}\hspace*{3em}}%
                % change 2em to the desired value
                \setlength{\leftmargin}{\labelwidth}%
                \addtolength{\leftmargin}{\labelsep}%
                \setlength{\parskip}{25pt}
                \setlength{\parindent}{25pt}
                \renewcommand{\makelabel}{\uplabel}
            }
      \fi}%
    {\ifAC@nolist%
     \else%
        \end{list}
     \fi}%
\makeatother

\usepackage{fancyhdr} % encabezado
\usepackage{graphicx} % imagenes
\usepackage{supertabular} % tablas
\usepackage{array}
\usepackage{color}
\usepackage{hhline}
\usepackage{setspace} %% For /singlespace
%%%%%\setlength{\parskip}{2mm plus 0.2mm minus 0.2mm}
\newcommand\arraybslash{\let\\\@arraycr}
\usepackage{relsize}  % tamaños relativos para el texto
\usepackage{ifthen}
\usepackage[center]{caption} %% Align caption images on center
\usepackage{url}
\usepackage{color}
\definecolor{gray97}{gray}{.97}
\definecolor{gray75}{gray}{.75}
\definecolor{gray45}{gray}{.45}

\usepackage{multirow}
\usepackage{colortbl}
\usepackage{color}
\usepackage{array}
\usepackage{tabularx}
\usepackage[table]{xcolor} %% Para \rowcolor
\usepackage{tabularx}
\usepackage{booktabs}

\definecolor{darkblue}{RGB}{79, 100, 128}
\definecolor{lila}{RGB}{240, 240, 240}
\definecolor{lilaover}{RGB}{196, 207, 219}


%% -- Fundiciones
\renewcommand{\rmdefault}{ptm}  % times
\renewcommand{\sfdefault}{lmss} % lmoderm sans sherif
\renewcommand{\ttdefault}{lmtt} % lmoderm typewriter

\usepackage[htt]{hyphenat}
%%\usepackage{atbeginend}
\usepackage{xifthen}

\usepackage{wrapfig}

\usepackage{subfig} % Image composition for figures

%% Párrafos
% La indentación que precede al párrafo normal debe ser de un
% cuadratín (1em) (-> 13.4.1.1)
% \setlength{\parindent}{1.7cm} %% Space for indent the beggining of a paragraph
\parindent   = 1em

% El espacio en blanco al final de una línea corta debe ser mayor que
% la indentación del párrafo. (-> 13.4.1.1)
\parfillskip = 3.5em plus 1fil

% añadir listados, bibliografia, etc a la tabla de contenido
\usepackage{tocbibind}

\usepackage{hvfloat}
\usepackage{adjustbox}
\usepackage{blindtext}

%% -- FORMATO DE CAPÍTULOS Y SECCIONES -----------------------------------------
\usepackage[rigidchapters, clearempty]{titlesec}
% - rigidchapters: Todos los títulos de capítulo tienen la misma altura
% - clearempty: Elimina encabezados y pies de páginas (izquierdas) vacías.

% espaciado entre secciones, subseciones, etc.
% doc: \titlespacing*{ command }{ left }{ beforesep }{ aftersep }[ right ]
\titlespacing{\section}{0pt}{5mm}{0mm}
\titlespacing{\subsection}{0pt}{4mm}{-1mm}
\titlespacing{\subsubsection}{0pt}{4mm}{-1mm}

\usepackage{tikz}

\newcommand{\chapterformat}[1]{%
	\begin{tikzpicture}
      \draw[fill,color=black] (0,0) rectangle (1cm,1cm);
      \draw[color=white] (0.5cm,0.5cm) node { \thechapter };
    \end{tikzpicture}
    \fontsize{14}{14}\selectfont\fontfamily{helvet} \textbf{\uppercase{#1}}
    
\hrulefill
\\
\newline
\newline


}



% doc: \titleformat{ command }[ shape ]{ format }{ label }{ sep }{ before }[ after ]

\newcommand{\frontchapterformat}{ 
  \titleformat{\chapter}[display]%
    {}
    {}
    {0mm}
    {\thechapter \chapter \chapterformat} 

  \titlespacing{\chapter}{0cm}{2cm}{2cm}
}

\let\frontmatterorig\frontmatter
\renewcommand{\frontmatter}{
  \frontmatterorig
  \singlespacing
  \pagestyle{fancy}
  \frontchapterformat
  \renewcommand{\chaptermark}[1]{\markboth{\uppercase{##1}}{}}
  \noindent\hrulefill

\noindent\hrulefill
}

\newcommand{\mainchapterformat}{
  \titleformat{\chapter}[display]
    {\normalfont\large\sffamily}
    {\chaptertitlename\ \thechapter}
    {0mm}
    {\chapterformat}
  \titlespacing{\chapter}{0cm}{1cm}{3.8cm}
}

\let\mainmatterorig\mainmatter
\renewcommand{\mainmatter}{
  %%\cleardoublepage
  \mainmatterorig
  \mainchapterformat
  %%\pagestyle{fancy}
  \renewcommand{\chaptermark}[1]{\markboth{\thechapter.\
  \textsc{##1}}{}}
  \onehalfspacing
  %%\cleardoublepage
}


\newcommand{\backchapterformat}{
  \frontchapterformat
}

\let\backmatterorig\backmatter
\renewcommand{\backmatter}{
  \backmatterorig
  \backchapterformat
  \cleardoublepage
}
% espaciado entre secciones, subseciones, etc.
% doc: \titlespacing*{ command }{ left }{ beforesep }{ aftersep }[ right ]
\titlespacing{\section}{0pt}{5mm}{0mm}
\titlespacing{\subsection}{0pt}{4mm}{-1mm}
\titlespacing{\subsubsection}{0pt}{4mm}{-1mm}


% Listar codigo fuente
%% Copiar carpeta tex en /usr/share/texmf/tex/
%% wget http://www.ctan.org/install/macros/latex/contrib/oberdiek.tds.zip
%%\usepackage{listingsutf8}[2007/11/11] 
\usepackage{listings}
%%\usepackage{listing}

% or code_color_gray.tex
\lstset{ frame=Ltb,
     framerule=0pt,
     aboveskip=0.3cm,
     belowskip=0.3cm,
     framextopmargin=3pt,
     framexbottommargin=3pt,
     framexleftmargin=0.4cm,
     framesep=0pt,
     rulesep=.4pt,
     backgroundcolor=\color{gray97},
     rulesepcolor=\color{black},
     %
     %%stringstyle=\ttfamily, %% Cadenas en negro
     stringstyle=\color[rgb]{0.627,0.126,0.941}, %% COLOR MORADO DE LAS CADENAS O STRINGS
     showstringspaces = false,
     basicstyle=\scriptsize\ttfamily,
     %% commentstyle=\color{gray45}, %% GRIS
     commentstyle=\color[rgb]{0.133,0.545,0.133}, %% COMENTARIOS VERDE CLARO
     %%keywordstyle=\bfseries, %% KEYWORDS EN NEGRITA
     keywordstyle=\color[rgb]{0,0,1}, %% KEYWORDS EN AZUL CLARO
     %
     numbers=left,
     numbersep=15pt,
     numberstyle=\tiny,
     numberfirstline = false,
     breaklines=true,
   }
 
% minimizar fragmentado de listados
\lstnewenvironment{listing}[1][]
   {\lstset{#1}\pagebreak[0]}{\pagebreak[0]}
 
\lstdefinestyle{consola}
   {basicstyle=\scriptsize\bf\ttfamily,
    backgroundcolor=\color{gray75},
   }
 
\lstdefinestyle{C}{language=C,}

\makeindex % Indice

%% David Villa: solución con \vspace
\newcommand{\pretitle}
{%
    \thispagestyle{empty}%
    \begin{center}%
        \vspace*{\stretch{3.2}}%
        {
            %% Imagen
            \begin{figure}[ht]
                \begin{center}
                    \includegraphics[scale=0.5]{src/img/escudo-esi.png} 
                \end{center}
            \end{figure}
            
            \fontsize{26}{26}
            \selectfont
            \begin{center}
                Universidad de Castilla-La Mancha
                Escuela Superior de Informática
            \end{center}
            
            % Subtitulo
            \fontsize{18}{18}
            \selectfont
            \textbf{{\csubtitle}}\\
            [.9cm]
            
            %Titulo
            \fontsize{22}{22}
            \selectfont
            \textbf{{\ctitle}}\\ 
            [1.5cm]
            \pfcdesc\\
            [1.5cm]
            
            % Fecha
            \fontsize{9}{9}
            \selectfont
             \textbf{\cdate}\\
            [4.5cm]
                
            % Autores
            \fontsize{12}{12}
            \selectfont
            \cauthor
            \manager
             
            
        \vspace*{\stretch{6.8}}%
        }
    \end{center}%
    \vfill
    \null
    %%\cleardoublepage
}

\newcommand{\putlogo}
{
    \includegraphics[width=35mm]{src/img/escudo-esi-monocromo.png}
}

\newcommand{\publishmonth}{}
\newcommand{\publishyear}{\year}

\newcommand{\publishdate}[2]{%
  \renewcommand{\publishyear}{#1}
  \renewcommand{\publishmonth}{#2}}
  
\newcommand{\frontpage}
{
  \pagestyle{empty}
  \thispagestyle{empty}
  \begin{center}
    \putlogo
    \vspace{25mm}\\
    {\Large \textbf{UNIVERSIDAD DE CASTILLA-LA MANCHA}} \\
    \bigskip
    {\Large \textbf{ESCUELA SUPERIOR DE INFORMÁTICA}} \\
    \vspace{25mm}
    {\Large \textbf{INGENIERÍA}} \\
    \bigskip
    {\Large \textbf{EN INFORMÁTICA}} \\
    \vspace{30mm}
    {\Large \textbf{PROYECTO FIN DE CARRERA}} \\
    \vspace{14mm}

    \begin{doublespace}
      {\Large FreeStation. Plataforma para el desarrollo de sistemas de
      distribución de software libre en puntos de información}
    \end{doublespace}

    \vspace{12mm}
    {\large Ángel Guzmán Maeso}\\
    \vfill
    \hfill
    {\large \textbf{\publishmonth, \publishyear}}
  \end{center}
  \pagestyle{plain} 
\thispagestyle{empty}
  \cleardoublepage
  \thispagestyle{empty}
  
  \pagenumbering{roman}
  \setcounter{tocdepth}{1}
  \setcounter{secnumdepth}{3} %% Sets the depth to number sections
  \setcounter{page}{1}
  
  \begin{center}
    \putlogo
    \vspace{15mm}\\
    {\Large \textbf{UNIVERSIDAD DE CASTILLA-LA MANCHA}} \\
    \bigskip
    {\Large \textbf{ESCUELA SUPERIOR DE INFORMÁTICA}}\\
    \vspace{14mm}
    {\Large Departamento de Tecnologías y Sistemas de Información}\\
    \vspace{45mm}
    {\Large \textbf{PROYECTO FIN DE CARRERA}} \\
    \bigskip

    \begin{doublespace}
      {\Large FreeStation. Plataforma para el desarrollo de sistemas de
      distribución de software libre en puntos de información}\\
    \end{doublespace}
  \end{center}

  \vspace{25mm}

  \begin{onehalfspace}
  \begin{flushleft}
    {\large Autor:\hspace{5mm} Ángel Guzmán Maeso}\\
    {\large Director: Carlos González Morcillo}\\
  \end{flushleft}
  \end{onehalfspace}
  
  \vfill
  \hfill
  {\large \textbf{\publishmonth, \publishyear}}
  
  \newpage
  
  \begin{minipage}[c][5cm][c]{1em}   % Introduce espacio en blanco efectivo
  \end{minipage}
  %\thispagestyle{empty}
  \newpage
}

%%\setlist{nolistsep}
\newenvironment{custom_itemize}{
\begin{itemize}
  \setlength{\itemsep}{1pt}
  \setlength{\parskip}{0pt}
  \setlength{\parsep}{0pt}
  \addtolength{\itemsep}{-0.20\baselineskip}
}{\end{itemize}}

\newenvironment{custom_enumerate}{
\begin{enumerate}
  \setlength{\itemsep}{1pt}
  \setlength{\parskip}{0pt}
  \setlength{\parsep}{0pt}
  \addtolength{\itemsep}{-0.20\baselineskip}
}{\end{enumerate}}

\setlength{\headheight}{16pt}

\newcommand{\address}[1]{\renewcommand{\address}{#1}}
\newcommand{\mcity}{Ciudad Real}
\newcommand{\city}[1]{\renewcommand{\mcity}{#1}}

\newcommand{\country}[1]{\renewcommand{\country}{#1}}
\newcommand{\phone}[1]{\newcommand{\mphone}{#1}}
\newcommand{\email}[1]{\newcommand{\memail}{#1}}
\newcommand{\homepage}[1]{\newcommand{\mhomepage}{#1}}

\newcommand{\license}{%
 %%\thispagestyle{empty}%
 %%\label{licensemini}
  \begin{minipage}{1.0\textwidth}
    \begin{singlespace}
      Permission is granted to copy, distribute and/or modify this
      document under the terms of the GNU Free Documentation License,
      Version 1.3 or any later version published by the Free Software
      Foundation; with no Invariant Sections, no Front-Cover Texts,
      and no Back-Cover Texts.  A copy of the license is included in
      the section entitled \hyperlink{chap:GFDL}{"GNU Free
        Documentation License"}.

      \smallskip
      Se permite la copia, distribución y/o modificación de este
      documento bajo los términos de la Licencia de Documentación
      Libre GNU, versión 1.3 o cualquier versión posterior publicada
      por la \emph{Free Software Foundation}; sin secciones
      invariantes. Una copia de esta licencia esta incluida en el
      apéndice titulado \hyperlink{chap:GFDL}{«GNU Free Documentation
        License»}.

    \smallskip
    Muchos de los nombres usados por las compañías para diferenciar
    sus productos y servicios son reclamados como marcas
    registradas. Allí donde estos nombres aparezcan en este documento,
    y cuando el autor haya sido informado de esas marcas registradas,
    los nombres estarán escritos en mayúsculas o como nombres propios.
  \end{singlespace}
  \end{minipage}
  %%\pageref{licensemini}
}

\newcommand{\jury}{%
%%\thispagestyle{empty}
  %%\vspace*{2cm}
  \cleardoublepage
  \noindent\underline{\textbf{{\Large TRIBUNAL:}}}
  \vspace{1.0cm}
  \par
  \textbf{{\large Presidente:}} \\ \par
  \textbf{{\large Vocal 1:}} \\ \par
  \textbf{{\large Vocal 2:}} \\ \par
  \textbf{{\large Secretario:}} \\ \par
  \vspace{2.2cm}
  \noindent\underline{\textbf{{\Large FECHA DE DEFENSA:}}} \par
  \vspace{2.2cm}
  \noindent\underline{\textbf{{\Large CALIFICACIÓN:}}} \par
  \vspace{3.6cm}
  \textbf{PRESIDENTE \hfil VOCAL 1 \hfil VOCAL 2 \hfil SECRETARIO} \par
  \vspace{3.1cm}
  Fdo.: \hfil Fdo.: \hfil Fdo.: \hfil Fdo.: \par
}

\newcommand{\copyrightpage}{%
\pagestyle{plain} 
%%\thispagestyle{empty}
\mbox{}
  \begin{singlespace}
    \pagestyle{plain} 
    \null \vfill \noindent
    \textbf{\cauthors} \par
    \smallskip \noindent
    \mcity\ -- \country \par
    \medskip \noindent
    \ifthenelse{\NOT \isundefined{\memail}}{\parbox{4em}{\emph{E-mail:}}\memail\\}{}
    %%\parbox{4em}{\emph{Teléfono:}}\mphone\\
    \parbox{4em}{\emph{Web site:}}{\small \url{\mhomepage}} \\
    
    \bigskip \noindent
    \copyright \ \year \cauthors \par
    \smallskip \noindent
    \pagestyle{plain} 
    \begin{minipage}{0.8\textwidth} \raggedright \footnotesize
      \license
    \end{minipage}
  \end{singlespace}
}

% -- símbolos
\usepackage{pifont}
\usepackage{tipa}


%% -- letras capitales
\usepackage{lettrine}
\newcommand{\drop}[2]{%
  \lettrine[lines=2,findent=2pt,nindent=3pt,loversize=0.1]% lhang=0.33
  {\textcolor[gray]{0.4}{#1}}{#2}%
  }
  

\newcommand{\dedication}[1]{%
  \null\vspace{\stretch{1}}
  \begin{flushright}
    \textit{#1}
  \end{flushright}
  \vspace{\stretch{2}}\null
}
 
\newcommand{\quoteauthor}[1]{\par\hfill#1\hspace{1em}\mbox{}\cleardoublepage}

\newcommand{\blankpage}
{
    \begin{titlepage}
		\fancyhf{}
		\newpage
		\mbox{}
	\end{titlepage}
}

%%\newcommand{\quoteauthor}[1]{\par\hfill#1\hspace{1em}\mbox{}}

% imprime: "Object Oriented (OO)"
\newcommand{\acx}[1]{\acused{#1}\acs{#1} %
  \nolinebreak[3] %
  (\acl{#1})}

\newcommand{\Acro}[2]{\acro{#1}{#2}\acused{#1}}
\newcommand{\sigla}[1]{\textsc{\textscale{.85}{#1}}}


\usepackage{tocbibind}


%%\setlength{\parskip}{4ex}
\setlength{\parskip}{20pt plus 10pt minus 1pt}

%% Load always as last package for avoid
%% Warning with footnotes: “name{Hfootnote.xx} has been referenced but does not
% exist”
\usepackage{hyperref} 
\usepackage{appendix}

\renewcommand{\appendixname}{Apéndices}
\renewcommand{\appendixtocname}{Apéndices}
\renewcommand{\appendixpagename}{Apéndices}




% Local Variables:
% coding: utf-8
% mode: latex
% TeX-master: "main"
% mode: flyspell
% ispell-local-dictionary: "castellano8"
% End:
% -*- coding: utf-8 -*-
%%\thispagestyle{empty}

\chapter*{Resumen}

\drop{L}{a} librenería o \acf{FS} es un software para centros 
o \acf{PAD} de información de software libre 
orientado a centros de enseñanza y universidades.

La posibilidad de una herramienta genérica para la distribución de software,
permite un gran abanico de posibilidades para extender de forma más sencilla
el uso del software libre en diferentes organizaciones e instituciones.

En la actualidad la mayoría de sistemas similares son propietarios y suelen
caer en la obsolescencia por la falta de personalizaciones propias a
determinados problemas.

La erradicación de estos problemas proponiendo una herramienta robusta y
configurable, cubre la futura demanda desde pequeñas a grandes empresas o
organismos.

Los repositorios de software modularizables configurados bajo la preferencia
del usuario, permiten a traves de un \acf{POI} un rápido acceso sin
complicaciones.

%%\thispagestyle{empty}
\newpage
\begin{minipage}[c][5cm][c]{1em}   % Introduce espacio en blanco efectivo
\end{minipage}
%\thispagestyle{empty}
\newpage

\chapter*{Abstract}

\drop{T}{he} librenería or \acf{FS} is a software for centers or \acf{APD} to
free software information oriented to teaching centers and universities.

The possibility of a generic tool for software distribution,
allows a wide range of possibilities to expand more easily
the use of free software in different organizations and institutions.

Currently most systems are proprietary and often similar
fall into obsolescence by the lack of customization specific to
certain problems.

The eradication of these problems by proposing a robust and
configurable, meet future demand from small to large companies or
agencies.

Modularized software repositories configured under the preference
the user, let through a \acf{POI} quick access without
complications.

%%\thispagestyle{empty}
\newpage
%%\thispagestyle{empty}
\begin{minipage}[c][5cm][c]{1em}   % Introduce espacio en blanco efectivo
\end{minipage}
%\thispagestyle{empty}
\newpage
%\thispagestyle{empty}

% Local Variables:
%   coding: utf-8
%   mode: latex
%   TeX-master: "main"
%   mode: flyspell
%   ispell-local-dictionary: "castellano8"
% End:

% -*- coding: utf-8 -*-
\mainmatter
\chapter{Introducción}
\thispagestyle{fancy}
\setcounter{page}{1}

\drop{L}{a}  distribución de la información ha sido un problema a resolver
desde antaño. Desde los primeros mensajes emitidos vía oral, pasando
por escritos hasta la forma más novedosa que ha logrado el ser humano, como
es internet, la complejidad relativa a las necesidades de comunicación ha ido en
aumento.

Con los nuevos tiempos, se ha hecho necesario transmitir esa información de
forma general y distribuida. La mayor parte de esta información es
procesada con ayuda de aplicaciones, cuya distribución y
configuración debe ser abordada previamente de forma específica a
nivel hardware y software con configuraciones generales y a cierto nivel de
detalle con personalizaciones.

La técnica inicial de distribución de software fue en principio únicamente
física. Tiendas especializadas en puntos muy concretos de un país, hacían
posible comprar una aplicación y usarla.
\acresetall
Con el avance de la tecnología, se ha conseguido poder descargar desde páginas
especializadas de forma directa mediante el protocolo \acs{FTP}\label{acro:FTP}
hasta redes \acs{P2P}\label{acro:P2P} empleando Bittorrent.

Normalmente estos modelos son generales y no se adaptan a las necesidades
específicas de personalización de ciertas organizaciones. Es necesario por
tanto, un modelo, en el que una organización o empresa pueda definir unos 
parámetros base generales y otros parámetros específicos para distribuir 
su software y configurarlo a medida de sus usuarios finales.

En esta solución, puede verse una gran utilidad para configuraciones de
\acs{POI}\label{acro:POI}s donde un usuario, pueda interactuar y conseguir
la distribución de software necesaria en un entorno determinado.

\newpage

\section{\uppercase{Qué es la distribución de software}}

La distribución del software vista como un servicio distribuido, necesita ser
rápida, escalable y robusta. Según ~\cite{LiB11} se trata de realizar
un modelo que se ajuste al software y a ningún otro requisito adicional, para
que la distribución sea tan sencilla y limpia que no sea posible cometer fallos
derivados. Este tipo de modelo general puede utilizarse en multitud de sectores,
como en el financiero, militares, médico, logístico, etc.

Fundamentalmente se trata de mejorar la forma de integrar y enviar datos y
aplicaciones, evitando cruzar los límites del costo del desarrollo de vida de
una aplicación, incluyendo el desarrollo, pruebas, integración, mantenimiento y
actualizaciones.

Los mensajes y aplicaciones enviadas por \acs{P2P} tienen una baja latencia y un
alto rendimiento, junto una alta disponibilidad, aunque pueden tener asociada
una pérdida de confiabilidad en los mensajes y suelen requerir un gran costo
de computación en servidores.
De este modo son una alternativa muy utilizada, cuando ciertos problemas
derivados de la pérdida de datos o ruido no son críticos en el dominio de
explotación.

Eliminando la complejidad de la conectividad punto a punto, se puede conseguir
un sistema escalable y robusto.

Para la distribución de software un aspecto clave es la interoperabilidad y la
\acf{QoS}. Garantizar la transmisión de cierta cantidad de
información en un tiempo determinado (\emph{throughput}) es una condición
relevante para un servicio de calidad. Este requisito resulta especialmente
importante en aplicaciones que requieren transmisión de vídeo o voz.

Un modelo dirigido por eventos o en tiempo real, permiten definir una
arquitectura orientada a servicios eficiente.

\newpage

\section{\uppercase{Introducción histórica}}

La distribución de software fue un gran reto no hace muchos años. En sus
inicios, un usuario necesitaba ser probablemente un experto en el
dominio\cite{ReJ12} para poder obtener una copia funcional del software 
requerido.

Por ejemplo, las primeras distribuciones GNU/Linux requerían tener conocimientos
avanzados en gestión de Sistemas, conociendo las bibliotecas y ejecutables
que eran necesarios para construir el sistema y conseguir finalmente un sistema
operativo funcional.

Los detalles específicos de configuración hacían
abandonar a muchas personas en esta tarea. Aunque el crecimiento de la comunidad
y el esfuerzo compartido llevaron a construir herramientas para construir sistemas
\acs{GNU}\label{acro:GNU}, más conocidas como las Autotools. Con una colección
de herramientas diseñadas para hacer la vida más fácil en la distribución de 
aplicaciones, los costes de tiempo se redujeron y se consiguió realizar
paquetes de aplicaciones más portables a otros sistemas compatibles con Unix.

Por tanto, se facilito al usuario final la posibilidad de recibir su propio
software empaquetado o realizar modificaciones sobre el mismo, habilitando
algunos parámetros sencillos en sus configuraciones.

La distribución de software en Berkeley con el sistema operativo BSD,
debe sus siglas a Berkeley Software Distribution. En las décadas de los setenta
y ochenta, la Universidad Berkeley se propuso compartir y
adaptar su software desde que la compañía ATT
retiró el permiso para usos comerciales. Debido a la demanda de un
gran público, el proceso de distribución de software fue una gran tarea a
considerar. Los usuarios demandaban personalizaciones y varias ramificaciones
del mismo proyecto surgieron como: OpenBSD, FreeBSD, SunOS, etc.
    
Estos hitos llevaron a hacer software escalable para un uso amplio. La
distribución del software empezó a convertirse en un conjunto de actividades
interrelacionadas que permitían producir o consumir sistemas necesarios para
producción.

La distribución por tanto incluye un proceso de obtención de software,
instalación, activación/desactivación, adaptación o actualización, integración,
seguimiento de cambios y desinstalación.

\section{\uppercase{Problemática}}

Los problemas derivados de la distribución en POI, conllevan grandes
requerimientos de ancho de banda entre servidor/es y posibles
clientes\cite{New10}. Si la red en la que se intenta distribuir cuenta con un
número elevado de usuarios, o se hace un uso intenso de los recursos puede
conllevar un mal funcionamiento del sistema\cite{Hel00}.

Los costes de los equipos suelen ser elevados (normalmente los
servidores) al igual que sus interconexiones.

Como problema derivado, el sistema necesita disponer de una determinada
homogeneización de componentes tanto hardware como software, para que habilite
una forma rápida y segura de hacer cambios de manera dinámica y evitar problemas
con diferentes versiones ejecutadas.

Las actualizaciones a menudo, requieren de personal cualificado con un buen
criterio estratégico que administren las prioridades necesarias para
que un software esté plenamente funcional con las últimas novedades sin
comprometer al sistema.

Las dificultades más comunes son la parametrización de atributos más adecuada
para el trabajo específico del usuario. El desarrollo para múltiples
plataformas, la interconexión robusta entre clientes y servidores y la
disponibilidad online del servicio y calidad del mismo.

\textbf{Solución general}

Las soluciones que se pueden aportar pasan por la difusión de software
empleando el método multicast \cite{Wil99} en el que se optimiza el ancho de
banda requerido del envío de la información.

Con esta solución, se optimiza la red con múltiples destinos, que reciben los
datos de forma simultánea con la estrategia más eficiente.

\newpage

Además se contempla implícitamente que el número de
incidencias recibidas en el centro de soporte anterior debe reducirse al
distribuir de forma más eficiente software específico vitales para la organización.

Para mitigar estos problemas, se propone el desarrollo de este Proyecto de Fin
de Carrera, FreeStation, cuyo objetivo general puede ser descrito como la
definición de una Plataforma para el desarrollo de sistemas de distribución de
software libre en puntos de información. La arquitectura modular y extensible de
\emph{FreeStation} permiten su cómoda adaptación a las necesidades específicas
de cada organización, facilitando al usuario final la obtención de software (o
datos) específicos a cada dominio de explotación específico.

El presente de Proyecto de Fin de Carrera debe basarse en una arquitectura
modular y extensible para añadir características nuevas o mejorar las ya
presentes.

Se entiende que poco a poco, de forma progresiva y con la ayuda del resultado de
estos primeros proyectos, se procederá a pensar en esta plataforma como medio
adicional de distribución de aplicaciones diseñadas a medida o de forma
generalizada, cuyo coste de mantenimiento y atención es muy inferior cuando se administra de modo
distribuido.

\section{\uppercase{Estructura del documento}}

Este documento se ha estructurado según las indicaciones de la normativa de
proyectos de fin de carrera de la Escuela Superior de Informática de la
Universidad de Castilla-La Mancha, empleando los siguientes capítulos:

\textbf{Capítulo 1: Introducción}

En este capítulo se ha realizado una breve introducción del ámbito en el que se encuadra
FreeStation.

\newpage

\textbf{Capítulo 2. Objetivos}

En este capítulo se desglosa y se describen la lista de objetivos y subobjetivos
planteados para este Proyecto de Fin de Carrera.

\textbf{Capítulo 3. Antecedentes}

En este capítulo se hace un repaso a los conocimientos y áreas que ha sido
necesario estudiar para el desarrollo de FreeStation, junto a los principales
sistemas existentes relacionados.

\textbf{Capítulo 4. Método de trabajo}

En este capítulo se explica y se justifica la metodología escogida para el
desarrollo de este sistema. Además se describen los recursos empleados, tanto
hardware como software.

\textbf{Capítulo 5. Arquitectura}

En este capítulo se describen los aspectos más importantes relativos al diseño e
implementación del sistema, detallando los problemas surgidos y las soluciones
aportadas. El capítulo se centra en la clasificación de los módulos y submódulos
que componen FreeStation, realizando una descripción ``\emph{top-down}'' desde un
enfoque funcional hasta un nivel de detalle técnico y específico de cada uno de
ellos.

\textbf{Capítulo 6. Evolución y costes}

En este capítulo se describe la evolución del sistema durante su período de
desarrollo, detallando las etapas e iteraciones realizadas.

Igualmente se aporta información relacionada con el coste económico, el análisis
de rendimiento (\emph{profiling}), y una serie de comparativas con aplicaciones
homólogas que no utilizan FreeStation, además de una encuesta realizada sobre su
uso.

\newpage

\textbf{Capítulo 7. Conclusiones y propuestas}

En este capítulo se hace un resumen a modo de conclusión del desarrollo y las
metas alcanzadas. También se enumera una serie de líneas de trabajo futuro planteadas
para continuar su desarrollo, y una conclusión personal.

\newpage
\thispagestyle{empty}